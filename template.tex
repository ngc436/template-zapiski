\documentclass{zapiski}
\originfo{529}{}{}{2023}
\setcounter{page}{177}
\date{12 октября 2023 г.}

%\usepackage[cp1251]{inputenc}
\usepackage[T1,T2A]{fontenc}
\usepackage[russian,english]{babel}

\usepackage{times}
\usepackage{url}
\usepackage{latexsym}
\usepackage{multirow}
\usepackage{lipsum}
\usepackage{graphicx}
\usepackage{amssymb}
\usepackage{hyperref}
\usepackage{booktabs}
\usepackage{tabularx}
\usepackage{caption}
\usepackage{cite}

\usepackage{paralist}
\usepackage{subcaption}
\usepackage{xcolor,colortbl}
\usepackage{array}
\usepackage{setspace}

\usepackage{placeins}

\usepackage{tikz}
\usepackage{pgfplots}
\pgfplotsset{width=10cm,compat=1.9}

\newcolumntype{H}{>{\setbox0=\hbox\bgroup}c<{\egroup}@{}}
\newcolumntype{Z}{>{\setbox0=\hbox\bgroup}c<{\egroup}@{\hspace*{-\tabcolsep}}}

\renewcommand{\UrlFont}{\ttfamily\small}
\newcommand{\cm}[1]{\textcolor{red}{#1}}
\newcommand{\bm}[1]{\textcolor{blue}{#1}}
\newcommand{\naacl}[1]{#1}

\usepackage{float}
\restylefloat{table}

\english

\begin{document}

\title[Your Paper Short Title]{Your Paper Full Title}


\author[Author]{Full Name}
\address[F.~Name]{Address}
\email{email}

\begin{abstract}
Abstract of your paper.
\end{abstract}

\keywords{key \and words.}

\maketitle

\section{Introduction}\label{sec:intro}

Below you could find a few samples of citations.

Hussain et al. \cite{hussain2017automatic} present a crowdsourced dataset of such advertisements,
including images and videos, and formulate several annotation tasks: topic detection,
sentiment type detection, symbolism recognition, strategy analysis, slogan annotation,
and Q/A for texts related to the ads' messages and motivation. We focus on the first three tasks,
each of which can be formulated as a classification problem.
A preliminary version of this work has appeared in \cite{savchenko2020ad}.

Fig.~\ref{fig:model} is a sample for image inclusion. Tab.~\ref{tab:ocr} is a sample of table inclusion.
Fig.~\ref{tab:topics_sentiments_overall_accuracy} is a sample of tikz figure inclusion.

\begin{figure}[!tbh]
    \centering
    \includegraphics[width=0.95\textwidth]{pic/model.png}
    \vspace{-.5cm}
    \caption{The proposed blending scheme.}
    \label{fig:model}\vspace{-.3cm}
\end{figure}


% \begin{table}[!tbh]\setlength{\tabcolsep}{2pt}
%     \begin{tabular}{p{.17\linewidth}|p{.79\linewidth}}
%         \hline
%         \emph{Tesseract}\newline  \cite{smith2007overview} & Maybe all that raven wants is to wet its beak in a cold glass of milk. poe mea lo EISsTo) aU eg
%         \\\hline
%         EAST+\newline\emph{Tesseract}\newline\cite{kopeykina2019automatic} & Maybe are Gein) eu SH Hostel S to wet its ake cold Me TS 10a milk. eee me glass 0) aa Penal see
%         \\\hline
%         PSENet \newline \cite{wang2019shape} &  Elle that eV WM eMey iy is 10 Wed Mey in beak | Ey milk. of (eek glass | ranlll@a a me al eee) glass be at jeffreycombs-com
%         \\\hline
%         EasyOCR \newline \cite{easyocr} &  Maybe all that raven poe me a glass of mik? beak in a cold glass ofmik. wants is to wet its seen 3t jefieycomlzesolm
%         \\\hline
%         Charnet \newline \cite{xing2019convolutional} & MAYBE ALL THAT RAVEN WANTS WET ITS BEAKIN COLD GLASS MILK POE GLASS MILK? SEEN ATJ COM
%         \\\hline
%         CloudVision \newline \cite{otani2018} & Maybe all that raven wants is to wet its beak in a cold glass of milk. poe me a glass of milk? seen at jeffreycombs.com
%          \\\hline
%     \end{tabular}

% \caption{Sample  texts obtained via OCR/captioning.}\label{tab:ocr}
% \end{table}


% \begin{figure}[!tbh]
% \centering
%     % \includegraphics[width=0.5\textwidth]{pic/Fig_ads_221_f1.eps}
%     \tikzset{dashdot/.style={dash pattern=on .4pt off 3pt on 4pt off 3pt}}
% \pgfplotsset{every tick label/.append style={font=\footnotesize}}
% \pgfplotsset{every axis legend/.append style={font=\footnotesize,draw=none}}
% \pgfplotsset{every axis label/.append style={font=\footnotesize}}

% \pgfplotsset{every axis/.append style={width=\linewidth, height={.6\linewidth}, grid=major, grid style={dashed,gray!25}, legend cell align=left, legend columns=1}}
% \pgfplotsset{every axis x label/.append style={at={(axis description cs:1.0,0.1)}, anchor=north east}}
% \pgfplotsset{every axis y label/.append style={at={(axis description cs:-0.,0.5)}, anchor=north}}

% \begin{tikzpicture}
%   \begin{axis}[
%     height=4cm,
%       xlabel={Threshold $t_0$},
%       ylabel={F1-score},
%       enlargelimits=false,
%       % ymin=1,
%       ytick={0,0.05,0.1,0.15},yticklabels={0,0.05,0.1,0.15},
%       % ytick={1,50,100},yticklabels={1,50,100},
%       xmax=1.0,xmin=0,
      % x tick label style={rotate=90,anchor=east} % Display labels sideways
    % ]
    % \addplot[color=blue,line width=0.9pt,mark size=1.5pt,mark=star] table[x index=0,y index=1] {pic/symbols_fscore_on_threshold.csv};
    % \addplot[color=red,line width=0.9pt,mark size=1pt,mark=*] table[x index=0,y index=2] {pic/symbols_fscore_on_threshold.csv};
    % \addplot[color=green!60!black,line width=0.9pt,mark size=1pt,mark=triangle] table[x index=0,y index=3] {pic/symbols_fscore_on_threshold.csv};
%     \legend{{MobileNet v1},{Inception v3},{EfficientNet-B3}};
%   \end{axis}
% \end{tikzpicture}

% 	\caption{F1-score for image-based symbol recognition (221 categories).}
% 	\label{visual-221-f1}


%     \footnotesize\setlength{\tabcolsep}{2.8pt}
%     \begin{tabular}{|c|c|c|}
%     \hline
%         CNN & Topics & Sentiments \\
%         \hline
%         Baseline~\cite{hussain2017automatic} & 60.34& 27.92\\
%         Curriculum learning~\cite{10.1007/978-3-030-01216-8_36} & --- & 27.96\\
%         \hline
%         ResNet-50 & 53.90 & 34.34 \\
%         Resnet-152 & 52.67 & 27.58 \\
%         Resnet-152 V2 & 52.12 & 27.64 \\
%         MobileNet v1& 50.56 & 33.50\\
%         MobileNet v2 & 54.76  & \textbf{34.58}\\
%         EfficientNet-B0 & 60.06 & 34.03\\
%         EfficientNet-B3 & \textbf{62.62} & 34.12 \\
%         \hline
%         Our multitask model & \textbf{62.99} & \textbf{36.27} \\ \hline
%     \end{tabular}
%     \captionof{table}{Image-based topic/sentiment classification.}
%     \label{tab:topics_sentiments_overall_accuracy}
% \end{figure}



\section*{Acknowledgments}
The work has been supported by the Fund grant XXX.

% \bibliographystyle{amsplain}
% \bibliography{anthology,eacl2021}

\providecommand{\bysame}{\leavevmode\hbox to3em{\hrulefill}\thinspace}
\providecommand{\MR}{\relax\ifhmode\unskip\space\fi MR }
% \MRhref is called by the amsart/book/proc definition of \MR.
\providecommand{\MRhref}[2]{
  \href{http://www.ams.org/mathscinet-getitem?mr=#1}{#2}
}
% \providecommand{\href}[2]{#2}
\begin{thebibliography}{10}
\end{thebibliography}
\end{document}